\documentclass{article}

\usepackage{ottalt}
\usepackage{mathpartir}
\usepackage{supertabular}

\usepackage{amsmath}
\usepackage{amssymb}

\usepackage{color}


%% Show admissible premises in rules
%% This should be false in main body of text and true in the appendix.
\newif\ifadmissible
\newcommand\suppress[1]{\ifadmissible{#1}\else{}\fi}
\inputott{qualitative-rules}

\title{System Specification}

\admissiblefalse
\begin{document}
\maketitle

This document is created directly from the definitions in the file
{\texttt{Qualitative.ott}}, with minor modifications as listed below.

This document is intended to specify, in a readable form, the subject of the
proofs of the paper as well as explain the slight differences
between this rendering, the paper, and the generated Coq files
\texttt{Qualitative\_ott.v} and \texttt{Qualitative\_inf.v}.

The reason for these slight differences is partly due to the restrictions of
the Ott locally nameless backend and the LNgen theory generation tool.
\begin{enumerate}
\item All parts of the syntax must be defined concretely in the Ott source file. 
\item All bound variables need to be explicitly determined.
\item All syntactic forms must bind at most one variable at a time.
\end{enumerate}

The first limitation is simply to accommodate through minor manual edits of
the outputs of Ott and LNgen. These edits allow us, for example, to
parameterize the development on an arbitrary semiring (see
\texttt{grade\_sig.v}) instead of working with a specific, concrete semiring.

The second limitation affects our generation of the typing rules for pattern
matching elimination forms, i.e. \textsc{T-Case} and \textsc{T-LetPair}
In these rules, we need to substitute in for the scrutinee
$y$ the result type $B$. 

The third limitation causes difficulty for the formalization of the
elimination rule for products. The usual pattern matching elimination
syntactic form binds two variables, one for each component of the tuple. This
is the form that is used in the submission. To accommodate Ott, in the
mechanization we replace the pattern matching elimination form for $\Sigma$
types with a slightly more general, but less familiar, form.

\section{System Specification}

\subsection{Grammar}

This language is parameterized over a lattics of grades, written $\ell$,
and a set of sorts, $\ottnt{s}$, which at must be nonempty.

\ottgrammartabular{
\otttm\ottinterrule
\ottcontext\ottafterlastrule
}

\subsection{Operational semantics}

\drules[ValueType]{}{Values that are types}{Type,Pi,WSigma,Sum,Unit}
\drules[V]{}{Values}{ValueType,TmUnit,WPair,InjOne,InjTwo}

\drules[S]{$ \ottnt{a}  \leadsto  \ottnt{a'} $}{Small-step operational semantics}{AppCong,Beta,CaseCong,CaseOneBeta,CaseTwoBeta,
  LetPairCong,LetPairBeta}

\subsection{Definitional equality}

\drules[CDefEq]{$ \Phi  \vdash  \ottnt{a}  \equiv_{ \ell }  \ottnt{b} $}{Conditional Definitional Equality}{Leq,Nleq}

\drules[Eq]{$ \Phi  \vdash  \ottnt{a}  \equiv_{ \ell }  \ottnt{b} $}{Definitional Equality}{Refl,Sym,Trans,Subst,Beta,Pi,Abs,App,PiFst,PiSnd,
    WSigma,WSigmaFst,WSigmaSnd,WPair,LetPair,
    Sum,SumFst,SumSnd,InjOne,InjTwo,Case,TyUnit,TmUnit}


\drules[CG]{$ \Phi  \vdash_{ k }^{ \ell }  \ottnt{a} $}{Conditional Grading}{Leq,Nleq}

\drules[G]{$ \Phi  \vdash_{ k }  \ottnt{a} $}{Grading}{Type,Var,Pi,Abs,App,WSigma,WPair,LetPair,
                             Sum,InjOne,InjTwo,Case,TyUnit,TmUnit}

\subsection{Type System}

As in pure type systems, this type system is parameterized by a set of axioms
($\ottkw{axiom} \, \ottnt{s_{{\mathrm{1}}}} \, \ottnt{s_{{\mathrm{2}}}}$) and rules ($\ottkw{rule} \, \ottnt{s_{{\mathrm{1}}}} \, \ottnt{s_{{\mathrm{2}}}} \, \ottnt{s_{{\mathrm{3}}}}$) that govern the treatment
of sorts.

\drules[CT]{$ \Omega \Vdash   \ottnt{a} :^{ \ell }  \ottnt{A} $}{Conditional Typing}{Leq,Top}

\drules[T]{$ \Omega  \vdash  \ottnt{a} :^{ \ell }  \ottnt{A} $}{Typing}{Type,Conv,Var,Pi,Abs,App,AppIrrel,WSigma,WPair,WPairIrrel,LetPairC,
              Sum,InjOne,InjTwo,CaseC,TmUnit,TyUnit}


\subsection{Auxiliary Judgements}

\drules[CEq]{$ \Phi  \vdash^{ \ell_{{\mathrm{0}}} }_{ \ell }  \ottnt{a}  \sim  \ottnt{b} $}{Conditional Graded Equality}{Leq,Nleq}
\drules[GEq]{$ \Phi  \vdash  \ottnt{a}  \sim_{ \ell }  \ottnt{b} $}{Graded Syntactic Equality}{Type,Var,Pi,Abs,App,WSigma,WPair,LetPair,Sum,InjOne,InjTwo,Case,TyUnit,TmUnit}

\drules[CPar]{$ \Phi  \vdash^{ \ell_{{\mathrm{0}}} }_{ \ell }  \ottnt{a}   \Rightarrow   \ottnt{b} $}{Conditional Parallel Reduction}{Leq,Nleq}
\drules[Par]{$ \Phi   \vdash    \ottnt{a}  \Rightarrow_{ \ell }  \ottnt{b} $}{Parallel reduction}{Refl,Pi,AppBeta,App,Abs,WSigma,WPair,WPairBeta,LetPair,
  Sum,InjOne,InjTwo,CaseBetaOne,CaseBetaTwo,Case}

\drules[MP]{$ \Phi   \vdash   \ottnt{a}   \Rightarrow^{\ast} _{ \ell }  \ottnt{b} $}{Parallel reduction, reflexive transitive closure}
   {Refl,Step}
\drules[]{$ \Phi \vdash   \ottnt{a}   \Leftrightarrow _{ \ell }   \ottnt{b} $}{Joinability}
   {join}


\end{document}
