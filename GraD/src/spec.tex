\documentclass{article}

\usepackage{ottalt}
\usepackage{mathpartir}
\usepackage{supertabular}

\usepackage{amsmath}
\usepackage{amssymb}

\usepackage{color}


%% Show admissible premises in rules
%% This should be false in main body of text and true in the appendix.
\newif\ifadmissible
\newcommand\suppress[1]{\ifadmissible{#1}\else{}\fi}
\inputott{dqtt-rules}

\title{System Specification}

\admissiblefalse
\begin{document}
\maketitle

This document is created directly from the definitions in the file
{\texttt{dqtt.ott}}, with minor modifications as listed below.

This document is intended to specify, in a readable form, the subject of the
proofs in Section 7.2 of the POPL paper ``A Graded Dependent Type SYstem with
a Usage-Aware Semantics'' as well as explain the slight differences between
the ott source file ({\texttt{dqtt.ott}}), this rendering, the POPL publication, and the
generated Coq files \texttt{dqtt\_ott.v} and \texttt{dqtt\_inf.v}.

The reason for these slight differences is due to the restrictions of the Ott
locally nameless backend and the LNgen theory generation tool. 
\begin{enumerate}
\item All parts of the syntax must be defined concretely in the Ott source file. 
\item All bound variables need to be explicitly determined
\item All syntactic forms must bind at most one variable at a time.
\end{enumerate}

The first limitation is simple to accommodate through minor manual edits of
the outputs of Ott and LNgen. These edits allow us to parameterize the
development on an arbitrary semiring (see \texttt{usage\_sig.v}) instead of
working with a specific, concrete semiring. 

The second limitation affects our generation of the typing rules for pattern
matching elimination forms, i.e. \textsc{T-UnitE}, \textsc{T-LetBox} and
\textsc{T-Case}. In these rules, we need to substitute in for the scrutinee
$y$ the result type $B$. Therefore, we need to inform Ott that $y$ should be
bound in $B$. We do so with an additional annotation (written
$\ottkw{@} \lambda y. B$) on the corresponding terms. For simplicity, in the
paper, we elide this annotation.

The third limitation causes difficulty for the formalization of the
elimination rule for tensor products. The usual pattern matching elimination
syntactic form binds two variables, one for each component of the tuple. This
is the form that is used in the POPL publication. To accommodate Ott, in this
version we replace the pattern matching elimination form for $\Sigma$ types
with a slightly more general, but less familiar, form.

We call this syntactic form ``spread'' and it has the following grammar.
\[
 \ottkw{spread}\,  \ottnt{a} \, \ottkw{to}\,  \ottmv{x} \, \ottkw{in}\,  \ottnt{b}\, \ottkw{@}\, \ottnt{B}
\]
This syntactic form binds the variable $x$ (corresponding to the first
component of the product) in the body $b$. The body $b$ must itself be a
function, where the argument is the second component of the tuple. By 
refactoring in this way, we observe Ott's restriction to single binding. 

Using this syntax, we can encode an elimination of an argument $a$
of type $ \Sigma  \ottmv{x} \!\!:^ \ottnt{q} \!\! \ottnt{A} . \ottnt{B} $, that uses
the usual pattern matching syntax
\[ 
     \ottkw{let}\, (\ottmv{x},\ottmv{y}) \,=\, (\ottnt{a} : \Sigma  \ottmv{x} \!\!:^ \ottnt{q} \!\! \ottnt{A} . \ottnt{B}) \ \ottkw{in}\  \ottnt{b} 
\] 

using the term
\[
   \ottkw{spread}\,  \ottnt{a} \, \ottkw{to}\,  \ottmv{x} \, \ottkw{in}\,  \lambda \ottmv{y} \!:^ \ottnt{q} \! \ottnt{A} . \ottnt{b}
   \ottkw{@}\, \lambda \ottmv{y} \!\:^\ottnt{q}\! \ottnt{A} . \ottnt{B}
\]

Even though $\ottkw{spread}$ solves the issue with single binding in the
syntax of the term, there is still the issue of generating its appropriate
elimination rule via Ott. Unfortunately, Ott Ott cannot express the correct
typing rule for $\ottkw{spread}$ because the typing rule requires a
substitution for both variables.  Therefore we modify the generated Coq
definition to include the appropriate substitution. For clarity, this document
includes the corresponding change in the typeset rule \textsc{T-Spread}.


\section{Grammar}

\ottgrammartabular{
\ottusage\ottinterrule
\otttm\ottinterrule
\ottcontext\ottinterrule
\ottD\ottafterlastrule
}


\section{Step relation}
\ottdefnsJOp{} 
\section{Typing relation}

\newcommand{\ottdruleTXXSpreadAlt}[1]{\ottdrule[#1]{%
\ottpremise{\ottnt{A}  \ottsym{=}   \Sigma  \ottmv{x} \!\!:^ \ottnt{q} \!\! \ottnt{A_{{\mathrm{1}}}} . \ottnt{A_{{\mathrm{2}}}} }%
\ottpremise{ \Delta ; \Gamma_{{\mathrm{1}}}  \vdash \ottnt{a} : \ottnt{A} }%
\ottpremise{  \Delta ,   \ottmv{x} \!\!:\!\! \ottnt{A_{{\mathrm{1}}}}   ;  \Gamma_{{\mathrm{2}}} ,   \ottmv{x} \!\!:^{ \ottnt{q} }\!\! \ottnt{A_{{\mathrm{1}}}}    \vdash \ottnt{b} :  \Pi  \ottmv{y} \!:^ \ottsym{1} \! \ottnt{A_{{\mathrm{2}}}} . \ottnt{B} \ottsym{\{}  (\ottmv{x},\ottmv{y})  \ottsym{/}  \ottmv{z}  \ottsym{\}} }%
\ottpremise{  \Delta ,   \ottmv{z} \!\!:\!\! \ottnt{A}   ;  \Gamma_{{\mathrm{3}}} ,   \ottmv{z} \!\!:^{ \ottnt{r} }\!\! \ottnt{A}    \vdash \ottnt{B} : \ottkw{type} }%
}{
 \Delta ; \Gamma_{{\mathrm{1}}}  \ottsym{+}  \Gamma_{{\mathrm{2}}}  \vdash  \ottkw{spread}\,  \ottnt{a} \, \ottkw{to}\,  \ottmv{x} \, \ottkw{in}\,  \ottnt{b}  : \ottnt{B}  \ottsym{\{}  \ottnt{a}  \ottsym{/}  \ottmv{z}  \ottsym{\}} }{%
{\ottdrulename{T\_Spread}}{}%
}}


\begin{ottdefnblock}[#1]{$ \Delta ; \Gamma  \vdash \ottnt{a} : \ottnt{A} $}{\ottcom{Typing}}
\ottusedrule{\ottdruleTXXsub{}}
\ottusedrule{\ottdruleTXXtype{}}
\ottusedrule{\ottdruleTXXvar{}}
\ottusedrule{\ottdruleTXXweak{}}
\ottusedrule{\ottdruleTXXpi{}}
\ottusedrule{\ottdruleTXXlam{}}
\ottusedrule{\ottdruleTXXapp{}}
\ottusedrule{\ottdruleTXXconv{}}
\ottusedrule{\ottdruleTXXunit{}}
\ottusedrule{\ottdruleTXXUnit{}}
\ottusedrule{\ottdruleTXXUnitE{}}
\ottusedrule{\ottdruleTXXBox{}}
\ottusedrule{\ottdruleTXXbox{}}
\ottusedrule{\ottdruleTXXletbox{}}
\ottusedrule{\ottdruleTXXsum{}}
\ottusedrule{\ottdruleTXXinjOne{}}
\ottusedrule{\ottdruleTXXinjTwo{}}
\ottusedrule{\ottdruleTXXcase{}}
\ottusedrule{\ottdruleTXXSigma{}}
\ottusedrule{\ottdruleTXXTensor{}}
\ottusedrule{\ottdruleTXXSpreadAlt{}}
\end{ottdefnblock}




\end{document}
