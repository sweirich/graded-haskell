\documentclass{article}

\usepackage{ottalt}
\usepackage{mathpartir}
\usepackage{supertabular}

\usepackage{amsmath}
\usepackage{amssymb}

\usepackage{color}


%% Show admissible premises in rules
%% This should be false in main body of text and true in the appendix.
\newif\ifadmissible
\newcommand\suppress[1]{\ifadmissible{#1}\else{}\fi}
\inputott{dqtt-rules}

\title{System Specification}

\admissiblefalse
\begin{document}
\maketitle

This document is created directly from the definitions in the file
{\texttt{dqtt.ott}}, with minor modifications listed below.

It is intended to specify, in a readable form, the syntactic type soundness
proof.

Note: there is one change here from the syntax shown in the paper. We replace
the pattern matching elimination form for $\Sigma$ types with a slightly more
general, but less familiar, form.

The reason for this change is that the Ott and LNgen tools limit language
specifications to single binding only. This prevents us from the usual
definition of the pattern matching elimination form for
$\Sigma$-types. Instead, we use an elimination form called ``spread'' of the
form
\[
 \ottkw{spread}\,  \ottnt{a} \, \ottkw{to}\,  \ottmv{x} \, \ottkw{in}\,  \ottnt{b}
\]
This syntactic form binds the variable $x$ (corresponding to the first
component of the product) in the body $b$. The body $b$ must itself be a
function, where the argument is the second component of the tuple.

In other words, we can encode an elimination of an argument $a$
of type $ \Sigma  \ottmv{x} \!\!:^ \ottnt{q} \!\! \ottnt{A} . \ottnt{B} $, that uses
the usual pattern matching syntax
\[ 
     \ottkw{let}\, (\ottmv{x},\ottmv{y}) \,=\, \ottnt{a} \ \ottkw{in}\  \ottnt{b} 
\] 

by using the term
\[
   \ottkw{spread}\,  \ottnt{a} \, \ottkw{to}\,  \ottmv{x} \, \ottkw{in}\,  \lambda \ottmv{y} \!:^ \ottnt{q} \! \ottnt{A} . \ottnt{b}
\]

\section{Grammar}

\ottgrammartabular{
\ottusage\ottinterrule
\otttm\ottinterrule
\ottcontext\ottinterrule
\ottD\ottafterlastrule
}


\section{Step relation}
\ottdefnsJOp{} 
\section{Typing relation}

Another issue with $\Sigma$ types is that Ott cannot express the complete
typing rule for $\ottkw{spread}$.  Therefore we need to modify the generate
Coq definition to include the appropriate substitution. This document includes the 
corresponding change in the typeset rule \textsc{T-Spread}.

\newcommand{\ottdruleTXXSpreadAlt}[1]{\ottdrule[#1]{%
\ottpremise{\ottnt{A}  \ottsym{=}   \Sigma  \ottmv{x} \!\!:^ \ottnt{q} \!\! \ottnt{A_{{\mathrm{1}}}} . \ottnt{A_{{\mathrm{2}}}} }%
\ottpremise{ \Delta ; \Gamma_{{\mathrm{1}}}  \vdash \ottnt{a} : \ottnt{A} }%
\ottpremise{  \Delta ,   \ottmv{x} \!\!:\!\! \ottnt{A_{{\mathrm{1}}}}   ;  \Gamma_{{\mathrm{2}}} ,   \ottmv{x} \!\!:^{ \ottnt{q} }\!\! \ottnt{A_{{\mathrm{1}}}}    \vdash \ottnt{b} :  \Pi  \ottmv{y} \!:^ \ottsym{1} \! \ottnt{A_{{\mathrm{2}}}} . \ottnt{B} \ottsym{\{}  (\ottmv{x},\ottmv{y})  \ottsym{/}  \ottmv{z}  \ottsym{\}} }%
\ottpremise{  \Delta ,   \ottmv{z} \!\!:\!\! \ottnt{A}   ;  \Gamma_{{\mathrm{3}}} ,   \ottmv{z} \!\!:^{ \ottnt{r} }\!\! \ottnt{A}    \vdash \ottnt{B} : \ottkw{type} }%
}{
 \Delta ; \Gamma_{{\mathrm{1}}}  \ottsym{+}  \Gamma_{{\mathrm{2}}}  \vdash  \ottkw{spread}\,  \ottnt{a} \, \ottkw{to}\,  \ottmv{x} \, \ottkw{in}\,  \ottnt{b}  : \ottnt{B}  \ottsym{\{}  \ottnt{a}  \ottsym{/}  \ottmv{z}  \ottsym{\}} }{%
{\ottdrulename{T\_Spread}}{}%
}}


\begin{ottdefnblock}[#1]{$ \Delta ; \Gamma  \vdash \ottnt{a} : \ottnt{A} $}{\ottcom{Typing}}
\ottusedrule{\ottdruleTXXsub{}}
\ottusedrule{\ottdruleTXXtype{}}
\ottusedrule{\ottdruleTXXvar{}}
\ottusedrule{\ottdruleTXXweak{}}
\ottusedrule{\ottdruleTXXdef{}}
\ottusedrule{\ottdruleTXXweakXXdef{}}
\ottusedrule{\ottdruleTXXpi{}}
\ottusedrule{\ottdruleTXXlam{}}
\ottusedrule{\ottdruleTXXapp{}}
\ottusedrule{\ottdruleTXXconv{}}
\ottusedrule{\ottdruleTXXunit{}}
\ottusedrule{\ottdruleTXXUnit{}}
\ottusedrule{\ottdruleTXXUnitE{}}
\ottusedrule{\ottdruleTXXBox{}}
\ottusedrule{\ottdruleTXXbox{}}
\ottusedrule{\ottdruleTXXletbox{}}
\ottusedrule{\ottdruleTXXsum{}}
\ottusedrule{\ottdruleTXXinjOne{}}
\ottusedrule{\ottdruleTXXinjTwo{}}
\ottusedrule{\ottdruleTXXcase{}}
\ottusedrule{\ottdruleTXXSigma{}}
\ottusedrule{\ottdruleTXXTensor{}}
\ottusedrule{\ottdruleTXXSpreadAlt{}}
\ottusedrule{\ottdruleTXXWith{}}
\ottusedrule{\ottdruleTXXPair{}}
\ottusedrule{\ottdruleTXXPrjOne{}}
\ottusedrule{\ottdruleTXXPrjTwo{}}
\end{ottdefnblock}




\end{document}
